\documentclass[a4paper,12pt]{article}
%\documentclass[a4paper,12pt]{scrartcl}

\usepackage[utf8x]{inputenc}
\usepackage{polski}

\title{Zadania domowe. Zestaw 3.1}
\author{Maciej Poleski}
\usepackage{amsmath}
\usepackage{amsfonts}
\usepackage{amssymb}
\usepackage{alltt}
\usepackage{listings}

\date{\today}

\pdfinfo{%
  /Title    (Zadania domowe. Zestaw 3.1)
  /Author   (Maciej Poleski)
  /Creator  (Maciej Poleski)
  /Producer (Maciej Poleski)
  /Subject  (ASD2)
  /Keywords (ASD2)
}

\begin{document}
\maketitle

\newpage

\section{}
Najbardziej odległa para punktów należy do otoczki wypukłej zbioru. Gdyby tak nie było, to moglibyśmy zwiększyć maksymalną odległość przedłużając odcinek łączący daną parę punktów aż do krawędzi otoczki (lub wierzchołka - wtedy bierzemy wierzchołek) i wybierając jeden z końców tej krawędzi (któryś z nich zwiększyłby odległość (bo oba punkty są wewnątrz otoczki)). Otoczke wypukłą możemy znaleźć np. algorytmem Grahama w czasie $n\log{n}$. Następnie przebiegamy dwoma wskaźnikami po otoczce wypukłej (zgodnie z ruchem wskazówek zegara) szukając pary najbardziej odległych punktów (przesuwamy pierwszy wskaźnik tak długo jak powoduje to zwiększenie odległości, po czym przesuwamy drugi i powtarzamy procedurę aż przejdziemy całą otoczke). Odpowiedzią jest para która w trakcie działania algorytmu miała największą odległość.

Taki spacer po otoczce jest liniowy.

\end{document}
