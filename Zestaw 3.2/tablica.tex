\documentclass[a4paper,12pt]{article}
%\documentclass[a4paper,12pt]{scrartcl}

\usepackage[utf8x]{inputenc}
\usepackage{polski}

\title{Zadania domowe. Zestaw 3.2}
\author{Maciej Poleski}
\usepackage{amsmath}
\usepackage{amsfonts}
\usepackage{amssymb}
\usepackage{alltt}
\usepackage{listings}

\date{\today}

\pdfinfo{%
  /Title    (Zadania domowe. Zestaw 3.2)
  /Author   (Maciej Poleski)
  /Creator  (Maciej Poleski)
  /Producer (Maciej Poleski)
  /Subject  (ASD2)
  /Keywords (ASD2)
}

\begin{document}
\maketitle

\newpage

\section{}
$
\begin{bmatrix}
 A & B & C \\
 1 & 0 & 0 \\
 0 & 0 & 1 \\
\end{bmatrix}
$
$\times$
$
\begin{bmatrix}
 x_{n-1} \\
 x_{n-2} \\
 1 \\
\end{bmatrix}
$
$=$
$
\begin{bmatrix}
 Ax_{n-1}+Bx_{n-2}+C \\
 x_{n-1} \\
 1 \\
\end{bmatrix}
$
$=$
$
\begin{bmatrix}
 x_{n} \\
 x_{n-1} \\
 1 \\
\end{bmatrix}
$


Korzystając z powyższych równości oraz z łączności mnożenia macierzy możemy ustalić poszukiwany wyraz ciągu podnosząc skrajnie lewą macierz do odpowiedniej ($k$) potęgi (algorytmem szybkiego potęgowania, nie zapominając o operacji modulo) i mnożąc tak uzyskaną macierz przez wektor (pionowy) [$x_1$ $x_0$ 1]. Pozostaje wybrać z uzyskanego wektora wynikowego środkową komórkę.

Rozmiar macierzy jest stały więc złożoność obliczeniowa jest złożonością algorytmu szybkiego potęgowania ($O(\log{k})$) (koszt mnożenia macierzy jest stały).

\section{}
Niech $A_k$ będzie macierzą taką że $a_{ij}^k$ = liczba ścieżek z wierzchołka i do j o długości k. Zauważmy że jeżeli $A$ jest macierzą sąsiedztwa to $A_{k+1} = A_k \times A$. Istotnie $a_{ij}^{k+1} = \sum_{l=1}^n a_{il}^{k}a_{lj}$, gdzie $a_{lj} = $ [istnieje krawędź między $l$ i $j$] ($l$ jest wierzchołkiem do którego docieramy w $k$ krokach i jeżeli tylko istnieje połączenie z $l$ do $j$, to składnik $a_{il}^k$ jest liczony do sumy).

Ponownie wykorzystujemy szybkie potęgowanie. Koszt mnożenia macierzy to $O(|V|^3)$ (z definicji). Koszt całego algorytmu to $O(|V|^3\log{k})$.

\end{document}
