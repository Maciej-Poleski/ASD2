\documentclass[a4paper,12pt]{article}
%\documentclass[a4paper,12pt]{scrartcl}

\usepackage[utf8x]{inputenc}
\usepackage{polski}

\title{Zadania domowe. Zestaw 1.2}
\author{Maciej Poleski}
\usepackage{amsmath}
\usepackage{amsfonts}
\usepackage{amssymb}
\usepackage{alltt}
\usepackage{listings}

\date{\today}

\pdfinfo{%
  /Title    (Zadania domowe. Zestaw 1.2)
  /Author   (Maciej Poleski)
  /Creator  (Maciej Poleski)
  /Producer (Maciej Poleski)
  /Subject  (ASD2)
  /Keywords (ASD2)
}

\begin{document}
\maketitle

\newpage

\section{}
\section{}
\section{}
Ciągom $p$ i $q$ będą odpowiadać wierzchołki grafu (każdy element ciągu to wierzchołek).
Tworzymy dodatkowo wierzchołki $s$ i $t$. Łączymy wierzchołek $s$ z każdym wierzchołkiem z ciągu $p$ krawędzią skierowaną o przepustowości równej odpowiednio $p_i$ (o ile jest niezerowa). Łączymy każdy wierzchołek z ciągu $q$ z wierzchołkiem $t$ krawędzią skierowaną o przepustowości odpowiednio $q_i$ (o ile jest niezerowa). Łączymy każdy wierzchołek z ciągu $p$ z każdym wierzchołkiem z ciągu $q$ krawędzią o przepustowości 1 (słownie: jeden). Uruchamiamy RTF. Odpowiedź brzmi tak jeżeli po zakończeniu algorytmu zostały wysycone wszystkie krawędzie wychodzące z $s$ i wszystkie krawędzie wchodzące do $t$. Jako bonus istnieje możliwość zrekonstruowania grafu dwudzielnego. $n+m<=2\max(n,m)$ więc algorytm ma złożoność $O(\max(n,m)^3)$.

\end{document}
