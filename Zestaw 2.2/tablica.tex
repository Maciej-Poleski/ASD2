\documentclass[a4paper,12pt]{article}
%\documentclass[a4paper,12pt]{scrartcl}

\usepackage[utf8x]{inputenc}
\usepackage{polski}

\title{Zadania domowe. Zestaw 2.2}
\author{Maciej Poleski}
\usepackage{amsmath}
\usepackage{amsfonts}
\usepackage{amssymb}
\usepackage{alltt}
\usepackage{listings}

\date{\today}

\pdfinfo{%
  /Title    (Zadania domowe. Zestaw 2.2)
  /Author   (Maciej Poleski)
  /Creator  (Maciej Poleski)
  /Producer (Maciej Poleski)
  /Subject  (ASD2)
  /Keywords (ASD2)
}

\begin{document}
\maketitle

\newpage

\section{}
\section{}
Palindrom nieparzysty $s$ wygląda tak, że jeżeli $s[0]$ jest środkiem, to $s[-i]\neq{s[i]}$ (dla wygody indeksuję od środka). Czyli tak naprawdę zmiana dotyczy sposobu porównywania znaków w Manacherze (mamy alfabet binarny). Możemy to sprawdzić za pomocą tej prostej i intuicyjnej implementacji (obliczającej równocześnie palindromy parzyste i nieparzyste).
\begin{alltt}
 T - tekst nad alfabetem skłądającym się z 2 znaków
 N - długość tekstu
 R - tablica długości 2*N+1
 
 k <- 0
 i <- 0
 while k<=2*N: 
    while (k-i)/2>0 and (k+i)/2<N and T[(k-i)/2-1]=T[(k+i)/2]:
      i <- i + 2
    R[k] <- i
    j <- 1
    while j<i and j+R[k-j]<i:
      R[k+j] <- R[k-j]
      j <- j + 1
    k <- k + j
    i <- max(i-j,k%2)
\end{alltt}
Na parzystych indeksach są długości kolejnych palindromów parzystych. Na nieparzystych nieparzystych. Można sobie przeszukać liniowo tablicę R...
\section{}
Mając słowo $a$ i $b$ (które być może jest prawie obrotem cyklicznym $a$) możemy przypatrzyć się słowu $bb$. W przypadku gdy jest to obrót cykliczny sytuacja wygląda tak: $a=xy$, $b=yx$ wtedy $bb=yxyx=yax$. W naszej sytuacji będziemy dopasowywać prefiks słowa $a$ do infiksu słowa $bb$ i sufiks $a$ do infiksu $bb$. Dodatkowo uwzględniamy fakt że między dopasowanym prefiksem a sufiksem musi być dokładnie jeden znak odstępu oraz łączna długość dopasowanego prefiksu + sufiksu + 1 musi być długością słowa $a$. Możemy wykorzystać prefikso-prefiksy (i sufikso-sufiksy) aby wyznaczyć dla każdej pozycji w słowie $bb$ długość najdłuższego prefiksu słowa $a$ zaczynającego się w tym miejscu. Analogicznie postępujemy dla sufiksów. Pozostaje teraz jednokrotnie przeszukać napis analizując wyznaczone tablice w celu uzyskania informacji, czy w danym miejscu istnieje satysfakcjonujące nas dopasowanie.

\end{document}
