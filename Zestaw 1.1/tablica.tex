\documentclass[a4paper,12pt]{article}
%\documentclass[a4paper,12pt]{scrartcl}

\usepackage[utf8x]{inputenc}
\usepackage{polski}

\title{Zadania domowe. Zestaw 1.1}
\author{Maciej Poleski}
\usepackage{amsmath}
\usepackage{amsfonts}
\usepackage{amssymb}
\usepackage{alltt}
\usepackage{listings}

\date{\today}

\pdfinfo{%
  /Title    (Zadania domowe. Zestaw 1.1)
  /Author   (Maciej Poleski)
  /Creator  (Maciej Poleski)
  /Producer (Maciej Poleski)
  /Subject  (ASD2)
  /Keywords (ASD2)
}

\begin{document}
\maketitle

\newpage

\section{}
Ścieżka $10^{40}$ krawędzi. Wyróżnione wierzchołki to wierzchołki o stopniu 1 na tej ścieżce. Przepustowości wszystkich krawędzi są takie same (np. 1). Każda krawędź wyznacza pewien minimalny przekrój. Wierzchołków jest $10^{40}+1$.
\section{}
\section{}
Algorytm E-K nadaje się do wyszukiwania minimalnego przekroju. Znaleziony minimalny przekrój ma minimalną ilość wierzchołków w zbiorze zawierającym źródło (pokazane na zajęciach). Można wykorzystać tą własność uruchamiając algorytm dwa razy. Za każdym razem rolę źródła i ujścia zamieniamy między wyróżnionymi wierzchołkami $x$ i $y$ (jeżeli graf jest skierowany to dodatkowo transponujemy go). Jeżeli za każdym razem uzyskamy ten sam podział to znaczy że istnieje tylko jeden minimalny przekrój. Jeżeli uzyskamy różne podziały - istnieje więcej niż jeden minimalny przekrój.
\section{}
Ustalamy przepustowość każdej krawędzi na jednostkową. Szukamy maksymalnego przepływu metodą Forda-Fulkersona. W efekcie znajdujemy (być może) kilka rozłącznych krawędziowo ścieżek od $x$ do $y$. Powiedzmy że ustalony maksymalny przepływ wynosi $a$. Każdą ścieżką wysyłamy $\lceil{s/a}\rceil$ szpiegów (lub mniej jeżeli do wysłania pozostało ich mniej). Trasy łatwo zrekonstruować przy użyciu sieci rezydualnej po zakończeniu metody F-F. Złożoność: szukanie ścieżki powiększającej $O(m)$ maksymalny przepływ w takim grafie $O(n)$ (co najwyżej ilość krawędzi wchodzących do ujścia - $O(n)$) wynik - $O(mn)$.
\section{}
Krawędzie o przepustowości jednostkowej. Tworzymy dodatkowy wierzchołek $a$ i łączymy z nim wierzchołki $y$ i $z$. Szukamy dwóch ścieżek powiększających z $x$ do $a$ (w czasie $O(n+m)$). Jeżeli znajdziemy - istnieją dwie rozłączne krawędziowo ścieżki łączące... to co trzeba. Jeżeli nie - to nie.

\end{document}
