\documentclass[a4paper,12pt]{article}
%\documentclass[a4paper,12pt]{scrartcl}

\usepackage[utf8x]{inputenc}
\usepackage{polski}

\title{Zadania domowe. Zestaw 4.2}
\author{Maciej Poleski}
\usepackage{amsmath}
\usepackage{amsfonts}
\usepackage{amssymb}
\usepackage{alltt}
\usepackage{listings}

\date{\today}

\pdfinfo{%
  /Title    (Zadania domowe. Zestaw 4.2)
  /Author   (Maciej Poleski)
  /Creator  (Maciej Poleski)
  /Producer (Maciej Poleski)
  /Subject  (ASD2)
  /Keywords (ASD2)
}

\begin{document}
\maketitle

\newpage

\section{}
\section{}
\section{}
\section{}
\section{}
\section{}
\section{}
C jest zbiorem klauzul (formuła wejściowa jest alternatywą klauzul należących do zbioru C). Każda klauzula jest zbiorem zmiennych (być może zanegowanych) (utożsamiamy ją z koniunkcją zmiennych należących do danej klauzuli).
\begin{alltt}
 for c in C:
    ok <- true
    for a in c:
        for b in c:
            if a jest sprzeczne z b:
                ok <- false
                break
    if ok:
        return true
 return false
\end{alltt}
 Zmienna $a$ jest sprzeczna z $b$ jeżeli wyrażenie $a=b$ jest równoważne wyrażeniu $x_i=\neg{x_i}$ dla pewnego $i$. (Czyli dotyczy tej samej zmiennej z tym że dokładnie jedna strona jest zanegowana). Jeżeli w chociaż jednej klauzuli nie ma sprzeczności to możemy wskazać wartościowanie nadając odpowiednie wartości kolejnym zmiennym z danej klauzuli (fałsz jeżeli zmienna jest zanegowana, prawda w przeciwnym wypadku). Pozostałe zmienne otrzymują dowolne wartościowanie. Algorytm jest kwadratowy (czyli wielomianowy).
 
 Błąd w przedstawionym rozumowaniu jest taki, że konwertując CNF do DNF możemy uzyskać formułę o wykładniczej długości (względem długości CNF). Oznacza to że nawet używając wielomianowego algorytmu dla DNF mamy czas wykładniczy względem długości CNF. Taki algorytm nie należy do $P$.
\end{document}
